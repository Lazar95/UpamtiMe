\part{Vizija sistema}

\chapter{Cilj dokumenta}
Cilj ovog dokumenta je da definiše zahteve koje treba da ispuni aplikacija \UpamtiOnline{}.



\chapter{Opseg dokumenta}
Dokument se odnosi na realizaciju veb aplikacije \UpamtiOnline{}.




\chapter{Reference}
%???




\chapter{Pozicioniranje proizvoda}

\section{Poslovne mogućnosti}
Primarni cilj aplikacije \UpamtiOnline{} je da omogući što adekvatniji proces učenja pojmova uz minimalno uloženo vreme, kao i da korisnicima stvori zabavnu i takmičarsku atmosferu.
Aplikacija treba da obezbedi efikasan sistem za učenje i obnavljanje gradiva, praćenje aktivnosti i napretka korisnika.
Takođe, treba obezbediti jednostavan i intuitivan način za proširenje opsega gradiva aplikacije kako bi korisnici, primenom istog sistema aplikacije, učili gradivo iz željenih oblasti.

\section{Postavka problema}

\begin{center}
    \begin{tabular}{|>{\columncolor[gray]{0.8}}m{2cm}|m{6cm}|}
        \hline
        Problem je &
        Nepostojanje dovoljno efikasnog sistema za učenje pojmova koji se teško pamte a lako zaboravljaju.
        Ovakvi pojmovi su, prvenstveno, reči stranih jezika, imena geografskih pojmova, datumi istorijskih događaja i slični.\\
        \hline
        Pogađa &
        Učenike osnovnih i srednjih škola, studente, ali i ostale ljude koji su zainteresovani da što brže i efikasnije nauče neki strani jezik ili gradivo iz bilo koje druge oblasti interesovanja.\\
        \hline
        Posledice su &
        Naporan i neefikasan proces savladavanja gradiva iz neke oblasti.
        Puno uloženog vremena ali nedovoljno rezultata.
        Konačno, gubljenje interesa i odustajanje od učenja.\\
        \hline
        Uspešno rešenje će &
        Učiniti proces učenja i pamćenja mnogo efikasnijim i zabavnijim uz znatno manje uloženog vremena i napora.\\
        \hline
    \end{tabular}
\end{center}

\section{Postavka pozicije proizvoda}

\begin{center}
    \begin{tabular}{|>{\columncolor[gray]{0.8}}m{2cm}|m{6cm}|}
        \hline
        Za koga &
        Sve ljude koji moraju ili žele da za što manje vremena nauče i u potpunosti savladaju gradivo iz neke oblasti, a pritom žele da proces bude što zabavniji.\\
        \hline
        Proizvod je &
        Veb-aplikacija.\\
        \hline
        Koja &
        Obezbeđuje efikasan sistem za učenje i obnavljanje pojmova kroz sesije koje će trajati ne više od nekoliko minuta, kao i praćenje aktivnosti i napretka korisnika.\\
        \hline
        Za razliku od &
        Postojećih rešenja kojih nema mnogo, nisu dovoljno efikasna ili zabavna.\\
        \hline
        Naš proizvod &
        Omogućuje lako deljenje kurseva za razliku od desktop aplikacija kao npr. Anki, pruža zabavnije i jednostavnije učenje novih pojmova.
        \\
        \hline
    \end{tabular}
\end{center}

\chapter{Opis korisnika}

\section{Opis potencijalnog tržišta}

\section{Profil korisnika}
Postoji samo jedan tip korisnika.
Svi korisnici se registruju i prijavljuju na isti način i imaju iste mogućnosti i privilegije u okviru aplikacije.

\section{Opis okruženja}
Kako je reč o veb-aplikaciji, bilo kom korisniku za rad neophodan je samo računar sa instaliranim veb-pregledačem i internet-konekcijom.
Na ovaj način, korisnik može da se prijavi na svoj nalog sa bilo kog računara koji poseduje pomenuto.

\section{Osnovne potrebe korisnika}
Korisnici treba da imaju mogućnost da uče i obnavljaju gradivo iz različitih oblasti interesovanja.
Takođe, korisnicima treba omogućiti jednostavno dodavanje novog gradiva koje će moći da uče korišćenjem istog sistema.

\section{Alternative i konkurencija}




\chapter{Opis proizvoda}

\section{Perspektiva proizvoda}

\section{Mogućnosti}

\begin{center}
    \begin{tabular}{|m{3cm}|m{7cm}|}
        \hline
        Koristi korisnika &
        Funkcije proizvoda koje to omogućavaju\\
        \hline
        Učenje pojmova &
        Stranica aplikacije koja sprovodi korisnika kroz sesiju učenja. Sesija obuhvata prikazivanje novih pojmova korisniku i čekanje na njegov odgovor, pritom nudeći mu različite tipove pomoći kako bi korisnik, korak po korak, što lakše savladao gradivo.\\
        \hline
        Obnavljanje pojmova &
        Analogno učenju, korisnik se sprovodi kroz sesiju obnavljanja dela gradiva za koje je već završio fazu učenja. U toku sesije, korisniku se prikazuju pojmovi i aplikacija određeno vreme čeka na odgovor od korisnika. U zavisnosti od tačnosti svakog pojedinačnog odgovora, pojmovi se odlažu određeno vreme, nakon koga će se smatrati da korisnik ponovo treba da ih obnovi.\\
        \hline
        Unos i ažuriranje gradiva &
        Panel aplikacije za kreiranje novih kurseva i unos pojmova omogućava korisniku da na sopstven način organizuje raspored lekcija po svojim kriterijumima i potrebama.\\
        \hline
        Pregled napretka i aktivnosti &
        Indeks stranica korisnika detaljno prikazuje skorašnje informacije o učenju i obnavljanju, kao i informacije o napretku korisnika koje prati.
        \\
        \hline
    \end{tabular}
\end{center}

\section{Pretpostavke i zavisnosti}
% ovde jedino da kazemo da radi lepo samo u chrome i firefox xDDD
\section{Cena}
Registrovanje naloga, kao i korišćenje svih usluga koje aplikacija pruža je skroz besplatno.
\section{Licenciranje i instalacija}
Aplikaciju nije potrebno instalirati, za korišćenje dovoljan je samo računar sa instaliranim modernim veb-pregledačem i internet-konekcijom.



\chapter{Funkcionalni zahtevi}

\section{Registrovanje korisnika}
Da bi koristio usluge sistema, svaki korisnik mora da bude registrovan, odnosno da ima kreiran nalog. Ovo je neophodno kako bi se identifikovao među ostalim korisnicima i kako bi bilo moguće pamtiti sve potrebne informacije vezane za tog korisnika.

\section{Prijavljivanje korisnika}
Kako bi informacije vezane za nekog registrovanog korisnika mogle da se ažuriraju, potrebno je da on bude prijavljen u trenutku korišćenja sistema.

\section{Učenje gradiva}
Učenja novih pojmova treba organizovati u kratke sesije koje će, na više različitih načina, upoznati korisnika sa gradivom i omogućiti mu da ga lako zapamti.

\section{Obnavljanje gradiva}
Obnavljanje gradiva takođe treba organizovati u sesije koje će biti dostupne korisniku neko vreme nakog učenja tog gradiva. U ovim sesijama se treba ustanoviti koje je pojmove korisnik znao odnosno zaboravio i u zavisnosti od toga treba organizovati buduće sesije obnavljanja.

\section{Kreiranje i ažuriranje kursa}
Ako korisnik ne može naći kurs koji bi hteo da pohađa ili nije zadovoljan gradivom koje je pokriveno, treba mu omogućiti da napravi svoj kurs. Neophodno je obezbediti lak i intuitivan način unosa i organizovanja lekcija i pojmova u okviru novog kursa. Takođe, treba omogućiti ispravljanje i dodavanje novog gradiva.

\section{Dodavanje prijatelja}
Svakom korisniku treba obezbediti mogućnost praćenja aktivnosti drugih korisnika sistema.

\chapter{Ograničenja}




\chapter{Zahtevi u pogledu kvaliteta}




\chapter{Prioritet funkcionalnosti}




\chapter{Nefunkcionalni zahtevi}

\section{Standardi}

\section{Sistemski zahtevi}
Ne zahtevaju se računari visokih performansi. Aplikaciju korisnik može da koristi sa bilo kog računara koji ima instaliran moderan veb-pregledač i uspostavljenu internet-konekciju.

\section{Performanse}
% mozda da se napise nesto samo kao da performanse ne zavise ni od cega posebno mozda eto samo da ne bude prespora internet konekcija xDD

\section{Okruženje}


% odavde verovatno treba maknemo stvari koje necemo da napisemo na kraju
\chapter{Dokumentacija}

\section{Korisničko upustvo}

\section{Online help} %TODO ovo ću da prevedem boli me kurac

\section{Upustvo za instalaciju i konfigurisanje}

\section{Pakovanje proizvoda}
