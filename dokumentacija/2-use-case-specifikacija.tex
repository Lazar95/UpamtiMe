\part{Specifikacija slučajeva korišćenja}

\chapter{Uvod}
Slučajevi korišćenja mogu biti podeljeni u sledeće grupe.
\begin{enumerate}
  \item Pristup aplikaciji:
  \begin{enumerate}
    \item Registracija.
    \item Prijava.
  \end{enumerate}
  \item Kursevi:
  \begin{enumerate}
    \item Pretraga kurseva.
    \item Prijava na kurs.
    \item Kreiranje novog kursa.
    \item Izmena kreiranog kursa.
    \item Navigacija kroz kurs.
    \item Pregled detaljne statistike.
  \end{enumerate}
  \item Sesije:
  \begin{enumerate}
    \item Učenje novih.
    \item Obnavljanje.
    \item Spajalica.
  \end{enumerate}
  \item Korisnici:
  \begin{enumerate}
    \item Pregled profila.
    \item Follow drugih profila.
    \item Pregled detaljnih statistika.
  \end{enumerate}
\end{enumerate}



\chapter{Akteri}
\label{ch:akteri}

\section{Korisnik}
\label{sec:korisnik}
Korisnik aplikacije može biti svako lice koje poseduje ispravan računar sa modernim veb-pretraživačem i internet-konekcijom.



\chapter{Slučajevi korišćenja}
\label{ch:slucajevi-koriscenja}

\section{Pristup aplikaciji}
\label{sec:pristup-aplikaciji}
Većina mogućnosti koje aplikacija pruža postaju dostupne korisniku tek nakon registracije, odnosno prijave.

Korisnik koji želi da u potpunosti koristi aplikaciju mora da se registruje na istu (\ref{subsec:registracija}), i po potrebi da se prijavi (\ref{subsec:prijava}).
Ukoliko korisnik trenutno nije prijavljen na aplikaciju, pri njenom pokretanju će dobiti mogućnost registracije i prijave.
Ukoliko je već prijavljen, pri pokretanju aplikacije korisnik vidi svoj profil (\ref{TODO}).

\subsection{Registracija}
\label{subsec:registracija}
Registracija se obavlja ispunjavanjem forme sa glavne strane aplikacije i pritiskom na dugme \dugme{registracija}.
Da bi se registracija uspešno obavila, neophodno je ispuniti sva ponuđena polja:
\begin{itemize}
  \item korisničko ime, \item šifra, \item ime i \item mejl adresa.
\end{itemize}

\emph{Korisničko ime} i \emph{mejl adresa} moraju biti jedinstveni u aplikaciji nezavisno jedni od drugog.
Drugim rečima, ne mogu postojati dva naloga sa istim adresama (čak i sa različitim korisničkim imenima), niti dva naloga sa istim korisničkim imenima (čak i sa različitim adresama).
Polja \emph{ime} i \emph{šifra} ne moraju biti jedinstveni za aplikaciju.

U slučaju da korisnik unese \emph{korisničko ime} i/ili \emph{mejl adresu} koja ne zadovoljava opisane uslove, dobija odgovarajuću poruku o tome.
Tek će se nakon ispunjenja svih uslova korisniku odobriti registracija.

\subsection{Prijava}
\label{subsec:prijava}
