\part{Specifikacija slučajeva korišćenja}
\label{part:use-case}

\chapter{Uvod}
\label{ch:use-case-uvod}
Slučajevi korišćenja mogu biti podeljeni u sledeće grupe.
\begin{enumerate}
  \item Pristup aplikaciji:
  \begin{enumerate}
    \item Registracija.
    \item Prijava.
  \end{enumerate}
  \item Kursevi:
  \begin{enumerate}
    \item Pretraga kurseva.
	\item Pregled gradiva kursa.
    \item Upis na kurs.
    \item Kreiranje novog kursa.
    \item Ažuriranje kursa.
    \begin{enumerate}
      \item Dodavanje lekcije.
      \item Ažuriranje lekcije.
    \end{enumerate}
    \item Navigacija kroz kurs.
    \item Pregled detaljne statistike.
  \end{enumerate}
  \item Sesije:
  \begin{enumerate}
    \item Učenje.
    \item Obnavljanje.
    \item Spajalica.
  \end{enumerate}
  \item Korisnici:
  \begin{enumerate}
    \item Praćenje korisnika.
    \item Pregled detaljnih statistika.
  \end{enumerate}
\end{enumerate}



\chapter{Akteri}
\label{ch:akteri}

\section{Korisnik}
\label{sec:korisnik}
Korisnik aplikacije može biti svako lice koje poseduje ispravan računar sa modernim veb-pretraživačem i internet-konekcijom.



\chapter{Slučajevi korišćenja}
\label{ch:slucajevi-koriscenja}

\section{Pristup aplikaciji}
\label{sec:pristup-aplikaciji}
Većina mogućnosti koje aplikacija pruža postaju dostupne korisniku tek nakon registracije, odnosno prijave.

Korisnik koji želi da u potpunosti koristi aplikaciju mora da se registruje na istu (\ref{subsec:registracija}), i po potrebi da se prijavi (\ref{subsec:prijava}).
Ukoliko korisnik trenutno nije prijavljen na aplikaciju, pri njenom pokretanju će dobiti mogućnost registracije i prijave.
Ukoliko je već prijavljen, pri pokretanju aplikacije korisnik vidi svoj profil (\ref{TODO}).

\subsection{Registracija}
\label{subsec:registracija}

\subsubsection{Kratak opis}
Pre početka korišćenja aplikacije korisnik se registruje unošenjem potrebnih informacija. Ovime, korisnik kreira svoj nalog koji će koristiti u daljem radu.

\subsubsection{Osnovni tok događaja}
\begin{itemize}
  \item Posećivanjem početne strane aplikacije pojavljuje se panel za registrovanje.
  \item Korisnik popunjava sva polja neophodna za registrovanje.
  \item Klikom na dugme \dugme{registracija} vrši se upis novog korisnika u bazu.
\end{itemize}

\subsubsection{Alternativni tokovi}
Ako se, nakon klika na dugme \dugme{registracija}, proverom validnosti unetih informacija ustanovi da je došlo do greške, korisnik će biti obavešten kako bi mogao da koriguje unete informacije.
Tek će se nakon ispunjenja svih uslova korisniku odobriti registracija.

\subsubsection{Preduslovi}
Da bi se registracija uspešno obavila, neophodno je ispuniti sva ponuđena polja:
\begin{itemize}
  \item korisničko ime, \item šifra, \item ime i \item mejl adresa.
\end{itemize}

\emph{Korisničko ime} i \emph{mejl adresa} moraju biti jedinstveni u aplikaciji nezavisno jedni od drugog.
Drugim rečima, ne mogu postojati dva naloga sa istim adresama (čak i sa različitim korisničkim imenima), niti dva naloga sa istim korisničkim imenima (čak i sa različitim adresama).
Polja \emph{ime} i \emph{šifra} ne moraju biti jedinstveni za aplikaciju.

\subsubsection{Posledice}
U slučaju uspešnog registrovanja, sistem automatski prijavljuje korisnika na novokreirani nalog i otvara njegovu indeks stranicu, u suprotnom se od korisnika traži da ponovi pokušaj registrovanja.


\subsection{Prijava}
\label{subsec:prijava}

\subsubsection{Kratak opis}
Ako korisnik ima nalog a nije prijavljen, korisnik se prijavljuje na svoj nalog pre korišćenja aplikacije.

\subsubsection{Osnovni tok događaja}
\begin{itemize}
  \item Posećivanjem početne strane aplikacije pojavljuje se panel za prijavljivanje.
  \item Korisnik popunjava sva polja neophodna za prijavljivanje.
  \item Ako želi, štiklira polje \emph{zapamti me}.
  \item Klikom na dugme \dugme{prijava} sistem pretražuje unete informacije u bazi i prijavljuje korisnika.
\end{itemize}

\subsubsection{Alternativni tokovi}
Ako se, nakon klika na dugme \dugme{prijavi}, proverom validnosti unetih informacija ustanovi da je došlo do greške, korisnik će biti obavešten kako bi mogao da koriguje unete informacije.
Tek će se nakon ispunjenja svih uslova korisniku odobriti prijavljivanje.

\subsubsection{Preduslovi}
Da bi se prijavljivanje uspešno obavilo, neophodno je ispuniti ponuđena polja \emph{korisničko ime} i \emph{šifra}.

\emph{Korisničko ime} mora postojati u bazi aplikacije i \emph{šifra} mora biti odgovarajuća kako bi se dozvolilo prijavljivanje korisnika.

\subsubsection{Posledice}
U slučaju uspešnog prijavljivanja korisniku će se otvoriti indeks stranica njegovog naloga, u suprotnom će se od korisnika tražiti da ponovi pokušaj prijavljivanja.





\section{Kursevi}
\label{sec:kursevi}
Kursevi su glavna jedinica organizacije gradiva.
Svim posetiocima treba obezbediti pretragu svih postojećih kurseva koji su organizovani u kategorije i detaljan pregled njihovog sadržaja.
Ako posetilac ima nalog i ako je prijavljen, ima mogućnost da se upiše na neki kurs i tako postane učesnik.
Svi prijavljeni korisnici takođe mogu da kreiraju svoje kurseve i da ih ažuriraju.

\subsection{Pretraga kurseva}
\label{subsec:pretraga-kurseva}

\subsubsection{Kratak opis}
Pretraživanje kurseva aplikacije po unetim kriterijumima.

\subsubsection{Osnovni tok događaja}
Korisnik, na stranici pregleda kurseva, unosi ime ili deo imena kursa koji hoće da nađe u polje \emph{naziv} i opciono bira kategoriju i potkategoriju u kojoj će se vršiti pretraga željenog kursa.
Nakon toga, klikom na dugme \SearchButton vrši se pretraga i prikazuje se lista kurseva koji odgovaraju unetim kriterijumima.

\subsubsection{Alternativni tokovi}

\subsubsection{Preduslovi}

\subsubsection{Posledice}
Kako na stranici pregleda kurseva korisnik može da vidi sve kurseve koji postoje u aplikaciji, nako pretrage lista prikaza se redukuje samo na kurseve koji zadovoljavaju kriterijume pretrage.


\subsection{Pregled gradiva kursa}
\label{subsec:pregled-gradiva-kursa}

\subsubsection{Kratak opis}
Evaluacija gradiva koje neki kurs pokriva.

\subsubsection{Osnovni tok događaja}
Korisnik, klikom na dugme \dugme{spisak} za odgovarajuću lekciju u kursu, otvara prozor koji prikazuje informacije vezane za tu lekciju.

\subsubsection{Alternativni tokovi}

\subsubsection{Preduslovi}

\subsubsection{Posledice}



\subsection{Upis na kurs}
\label{subsec:upis-na-kurs}

\subsubsection{Kratak opis}
Korisnik se upisuje na kurs kako bi mogao da uči i obnavlja gradivo koje taj kurs sadrži.

\subsubsection{Osnovni tok događaja}
Na stranici kursa klikom na dugme \dugme{upiši se}, korisnik se upisuje na dati kurs, odnosno postaje učesnik tog kursa.

\subsubsection{Alternativni tokovi}

\subsubsection{Preduslovi}
Korisnik mora biti prijavljen kako bi mu se na stranici kursa prikazalo dugme \dugme{upiši se}.

\subsubsection{Posledice}
Nakon upisa na kurs, korisniku su omogućene opcije za učenje i obnavljanje lekcija tog kursa.



\subsection{Kreiranje kursa}
\label{subsec:kreiranje-kursa}

\subsubsection{Kratak opis}
Korisnik kreira novi kurs koji će samo on moći da ažurira.

\subsubsection{Osnovni tok događaja}
\begin{itemize}
  \item Korisnik unosi naziv kursa koji će da kreira i kategoriju kojoj će taj kurs da pripada.
  \item Ako želi i ako odgovarajuća potkategorija postoji, može izabrati i potkategoriju.
  \item Klikom na dugme \dugme{kreiraj} vrši se upis potrebnih informacija u bazu.
\end{itemize}

\subsubsection{Alternativni tokovi}
Ako korisnik nije uneo naziv kursa ili nije odabrao kategoriju, dobiće odgovarajuću poruku.

\subsubsection{Preduslovi}

\subsubsection{Posledice}
Nakon uspešnog kreiranja novog kursa, automatski se otvara stranica za ažuriranje tog kursa.


\subsection{Ažuriranje kursa}
\label{subsec:azuriranje-kursa}

\subsubsection{Kratak opis}
Kreator kursa vrši ažuriranje informacija koje su vezane za kurs i za lekcije koje pripadaju tom kursu.
Nakon unošenja svih potrebnih izmena potrebno je kliknuti na dugme \SaveButton{} čime se završava ažuriranje.

\subsubsection{Osnovni tok događaja}
\begin{itemize}
  \item Na stranici kursa, kreator klikom na dugme \EditCourseButton{} otvara stranicu za ažuriranje kursa.
  \item Klikom na ime kursa omogućeno je kucanje novog imena. Nakon izmene, klikom na \CheckButton se potvrdjuje promena imena. Klikom na \CrossButton novo ime se odbacuje i kurs zadržava prvobitno ime.
  \item Klikom na dugme \EditCourseButton{} za opis kursa moguće je ažurirati njegov opis. Nakon izmene, klikom na \CheckButton se potvrdjuje promena opisa. Klikom na \CrossButton novi opis se odbacuje i kurs zadržava prvobitni opis.
  \item Klikom na dugme \ChangeAvatarButton{} otvara se standardni meni za izbor nove slike. %ovde možda da se doda ono klikom na choose file pa na upload pa bla bla
  \item Nakon željenih izmena, klikom na dugme \SaveButton{} se potvrdjuju unete izmene i nove informacije o tom kursu su sada sačuvane.
\end{itemize}
\subsubsection{Alternativni tokovi}
Ako nakon izmena nove informacije nisu validne, klikom na dugme \SaveButton{} neće biti ažurirane informacije o kursu i korisniku će se prikazati poruka kako čuvanje novih informacija nije bilo uspešno.
\subsubsection{Preduslovi}
Da bi korisnik uopšte mogao da pristupi stranici za ažuriranje kursa, on mora biti kreator tog kursa.
\subsubsection{Posledice}
Nove informacije o kursu koje je korisnik uneo se čuvaju u bazi i ovime je završen proces ažuriranja.

\subsection{Kreiranje lekcije}
\label{subsec:kreiranje-lekcije}

\subsubsection{Kratak opis}
Korisnik dodaje novu lekciju u neki od svojih kurseva.

\subsubsection{Osnovni tok događaja}
\begin{itemize}
  \item Na stranici kursa, kreator klikom na dugme \PlusButton{} započinje pravljenje nove lekcije.
  \item Korisnik zatim unosi ime nove lekcije.
  \item Ponovnim klikom na dugme \PlusButton{} potvrđuje kreiranje lekcije sa unetim imenom.
\end{itemize}
\subsubsection{Alternativni tokovi}
Ako korisnik nije uneo naziv lekcije ili lekcija sa unetim nazivom već postoji u okviru tog kursa, dobiće odgovarajuću poruku.
\subsubsection{Preduslovi}
Da bi kreiranje lekcije bilo uspešno, u okviru tog kursa ne sme već postojati lekcija sa istim imenom.
\subsubsection{Posledice}
Nakon uspešnog kreiranja, nova lekcija se automatski dodaje na listu postojećih lekcija tog kursa i može se vršiti njeno ažuriranje.


\subsection{Ažuriranje lekcije}
\label{subsec:azuriranje-lekcije}

\subsubsection{Kratak opis}
Korisnik vrši izmenu informacija o samoj lekciji ili o pojmovima koji se uče unutar te lekcije.
Nakon unošenja svih potrebnih izmena potrebno je kliknuti na dugme \SaveButton{} čime se završava ažuriranje.

\subsubsection{Osnovni tok događaja}
\begin{itemize}
  \item Klikom na dugme \SettingsButton{} otvara se padajući meni koji nudi više opcija za ažuriranje:
  \begin{itemize}
    \item \emph{Promeni ime} dozvoljava korisniku da unese novo ime lekcije.
    Novo ime se potvrdjuje klikom na \CheckButton{} ili poništava klikom na \CrossButton{}.
    \item \emph{Obriši lekciju} u potpunosti briše sve informacije vezane za tu lekciju kao i za sve pojmove unutar lekcije.
    \item \emph{Grupno menjanje} dozvoljava korisniku da ažurira informacije o bilo kom pojmu u toj lekciji.
    Sve izmene se istovremeno potvrđuju klikom na \CheckButton{} ili poništavaju klikom na \CrossButton{}.
    \item \emph{Zameni pitanje i odgovor} će, za svaki pojam unutar te lekcije, međusobno zameniti informacije zapisane u poljima \emph{pitanje} i \emph{odgovor}.
    \item \emph{Promeni opis svima} omogućava korisniku da istovremeno promeni opis svim pojmovima unutar te lekcije. % ovde da se doda nešto ako treba kad ga laza promeni
  \end{itemize}
  \item Klikom na sličicu pored imena nivoa otvara se padajući meni različitih sličica i boja koje korisnik može odabrati.
  \item Dodavanje novog pojma započinje klikom na \PlusButton{} u okviru lekcije.
  Zatim se vrši unos informacija u polja \emph{pitanje}, \emph{odgovor} i \emph{opis}, a zatim ponovnim klikom na \PlusButton{} se potvrđuje dodavanje novog pojma.
  \item Ažuriranje postojećeg pojma se vrši klikom na \EditCourseButton za odgovarajući pojam.
  Nakon izmene željenih informacija korisnik potvrđuje izmene klikom na \CheckButton{} ili ih poništava klikom na \CrossButton{}.
  \item Brisanje postojećeg pojma se vrši klikom na \DeleteButton{} za odgovarajući pojam.
  Nakon ovoga, ako korisnik želi da poništi brisanje, to može uraditi klikom na \UndoButton{}.
  \item Kada korisnik završi sa ažuriranjem lekcije, klikom na \SaveButton{} potvrđuje sve izmene i baza se ažurira.
\end{itemize}
\subsubsection{Alternativni tokovi}
Ako informacije vezane za lekciju nakon izmena nisu validne, klikom na dugme \SaveButton{} ažuriranje baze se neće izvršiti i korisnik će biti obavešten odgovarajućom porukom.
\subsubsection{Preduslovi}

\subsubsection{Posledice}
Nove informacije o lekciji koje je korisnik uneo se čuvaju u bazi i ovime je završen proces ažuriranja.

\subsection{Navigacija kroz kurs}
\label{subsec:navigacija-kroz-kurs}

\subsubsection{Kratak opis}
Akcije koje korisnik može započeti u okviru kursa.
\subsubsection{Osnovni tok događaja}
\begin{itemize}
  \item Klikom na dugme \dugme{uči} korisnik započinje sesiju učenja podrazumevanog broja novih pojmova.
  Opciono, korisnik može klikom na padajući meni na istom dugmetu da promeni broj pojmova koje želi da uči u jednoj sesiji.
  \item Klikom na dugme \dugme{obnovi} korisnik započinje sesiju obnavljanja podrazumevanog broja pojmova.
  Opciono, korisnik može klikom na padajući meni na istom dugmetu da promeni broj pojmova koje želi da obnavlja u jednoj sesiji.
  \item Klikom na dugme \dugme{spajalica} korisnik započinje sesiju igre \emph{spajalica}.
\end{itemize}
\subsubsection{Alternativni tokovi}

\subsubsection{Preduslovi}
Korisnik mora biti prijavljen i mora biti učesnik na tom kursu kako bi imao ponuđene pomenute opcije.
\subsubsection{Posledice}



\subsection{Učenje}
\label{subsec:ucenje}

\subsubsection{Kratak opis}
Korisnik prolazi kroz sesiju učenja pojmova iz nekog kursa.

\subsubsection{Osnovni tok događaja}
\begin{itemize}
  \item
  Klikom na dugme \dugme{uči} korisnik započinje sesiju učenja podrazumevanog broja novih pojmova.
  Na dugme \dugme{uči} može se kliknuti na više različitih mesta:
  \begin{itemize}
    \item
    Ako je reč o dugmetu za određeni kurs na indeks stranici korisnika, započeće sesiju učenja podrazumevanog broja novih pojmova iz prve nenaučene lekcije tog kursa.
    \item
    Ako se korisnik nalazi na stranici kursa i klikne na dugme \dugme{uči} ispod slike tog kursa, započeće sesiju učenja podrazumevanog broja novih pojmova iz prve nenaučene lekcije tog kursa.
    \item
    Korisnik, takođe na stranici kursa, može kliknuti na lekciju koju želi da uči i onda klikom na dugme \dugme{uči} u okviru te lekcije započinje sesiju učenja podrazumevanog broja novih pojmova iz te konkretne lekcije.
  \end{itemize}
  Opciono, u drugom i trećem slučaju, korisnik može klikom na padajući meni na istom dugmetu da promeni broj pojmova koje želi da uči u jednoj sesiji.
  Pojmovi se prikazuju na više različitih načina, u zavisnosti od toga po koji put se taj pojam prikazuje u toku sesije.
  Načini prikaza su sledeći:
  \begin{itemize}
    \item
    Ako se određeni pojam prikazuje prvi put, korisniku se prikazuju dato pitanje i odgovor.
    Može kliknuti na dugme \dugme{nauči} ili na dugme \dugme{već znam}.
    Ako korisnik ne zna odgovor na dati pojam i želi da ga nauči, treba kliknuti na \dugme{nauči} ili pritisnuti taster \emph{enter}.
    U ovom slučaju, odabrani pojam će se pojaviti na više različitih načina u toku sesije kako bi ga korisnik naučio.
    Klikom na dugme \dugme{već znam}, dati pojam će se korisniku pojaviti samo još jednom u toku sesije.
    \item
    Kada se pojam prikaže drugi put, korisniku će biti prikazano pitanje i ponuđena četiri odgovora, od kojih je samo jedan tačan.
    Klikom na odgovarajući pojam ili pritiskom na taster \emph{1}, \emph{2}, \emph{3} ili \emph{4} na tastaturi, korisnik bira neki od odgovora.
    U slučaju da je odgovor bio tačan, određeni pojam se više neće prikazati na isti način u okviru te sesije.
    Ako je odgovor bio netačan, prikazuju se izabrani odgovor i onaj koji je tačan kako bi korisnik mogao da vidi zbog čega je pogrešio i zapamti tačan odgovor.
    Nakon toga, korisnik nastavlja sa sesijom klikom na \emph{dalje} ili pritiskom na taster \emph{enter}.
    Određeni način prikaza datog pojma na koji je unet pogrešan odgovor će se opet prikazati u toku sesije.
    \item
    Treći put se, za odgovarajući pojam, prikazuje pitanje i samo deo odgovora.
    Koristeći prikazani deo odgovora kao podsetnik, korisnik treba za dati pojam da unese odgovor sa tastature.
    Klikom na dugme \dugme{proveri odgovor} ili pritiskom na taster \emph{enter}, korisnik potvrđuje svoj odgovor.
    Ako je odgovor bio tačan, određeni pojam se više neće prikazati na isti način u okviru te sesije.
    U slučaju da nije tačan, prikazuju se uneti odgovor i onaj koji je tačan kako bi korisnik mogao da vidi zbog čega je pogrešio i zapamti tačan odgovor.
    Nakon toga, korisnik nastavlja sa sesijom klikom na dugme \emph{dalje} ili pritiskom na taster \emph{enter}.
    Ovaj način prikaza datog pojma na koji je unet pogrešan odgovor će se opet prikazati u toku sesije, ali ovaj put će deo odgovora koji je prikazan biti veći.
    \item
    Sledeći put kada se pojavi isti pojam, prikazuje se pitanje i skup ponuđenih karaktera.
    Skup ponuđenih karaktera sadrži sve karaktere koji čine tačan odgovor i određeni broj karaktera koji je višak.
    Korisnik treba da, klikom na određene karaktere ili korišćenjem tastature, unese tačan odgovor, karakter po karakter.
    Isto kao ranije, pritiskom na taster \emph{enter} ili klikom na dugme \dugme{proveri odgovor}, korisnik potvrđuje uneti odgovor.
    Ako je odgovor bio tačan, određeni pojam se više neće prikazati na isti način u okviru te sesije.
    U slučaju da nije tačan, prikazuju se uneti odgovor i onaj koji je tačan kako bi korisnik mogao da vidi zbog čega je pogrešio i zapamti tačan odgovor.
    Nakon toga, korisnik nastavlja sa sesijom klikom na dugme \emph{dalje} ili pritiskom na taster \emph{enter}.
    Određeni način prikaza datog pojma na koji je unet pogrešan odgovor će se opet prikazati u toku sesije, ali ovaj put će biti manje ponuđenih karaktera, i dalje zadržavajući one koji su neophodni za odgovor.
    \item
    Kada se pojam prikaže peti i poslednji put u okviru jedne sesije, korisniku se prikazuje samo pitanje.
    Korisnik unosi odgovor korišćenjem tastature i pritiskom na taster \emph{enter} ili klikom na dugme \dugme{proveri odgovor}, korisnik potvrđuje uneti odgovor.
    Ako je odgovor bio tačan, određeni pojam se više neće prikazati u okviru te sesije ni na jedan način.
    U slučaju da nije tačan, prikazuju se uneti odgovor i onaj koji je tačan kako bi korisnik mogao da vidi zbog čega je pogrešio i zapamti tačan odgovor.
    Nakon toga, korisnik nastavlja sa sesijom klikom na dugme \emph{dalje} ili pritiskom na taster \emph{enter}.
    Određeni način prikaza datog pojma na koji je unet pogrešan odgovor će se opet prikazati u toku sesije.
  \end{itemize}
  \item
  U zavisnosti od tačnosti odgovora koje je korisnik uneo, određeni pojmovi će biti dostupni za obnavljanje nakon određenog vremena.
  Ako je odgovor bio tačan, dati pojam će biti odložen za kasnije vreme u odnosu na onaj na koji je dat netačan odgovor.
  Kada korisnik odgovori na sve pojmove odgovarajući broj puta, sesija se završava i prikazuje se prozor kraja sesije.
  Klikom na dugme \dugme{nastavi}, započinje se nova sesija učenja, a klikom na dugme \dugme{nazad na kurs}, korisniku se prikazuje stranica kursa.
\end{itemize}
\subsubsection{Alternativni tokovi}

\subsubsection{Preduslovi}
Da bi mogao da pokrene sesiju učenja, korisnik treba da bude učesnik na tom kursu, odnosno da bude prijavljen na kurs.
Takođe, ako za konkretni kurs ili lekciju nema pojmova za učenje, neće se započeti sesija.
\subsubsection{Posledice}




\subsection{Obnavljanje}
\label{subsec:obnavljanje}

\subsubsection{Kratak opis}
Korisnik prolazi kroz sesiju učenja pojmova iz nekog kursa.

\subsubsection{Osnovni tok događaja}
\begin{itemize}
  \item
  Klikom na dugme \dugme{obnovi} korisnik započinje sesiju obnavljanja podrazumevanog broja naučenih pojmova.
  Na dugme \dugme{obnovi} može se kliknuti na više različitih mesta:
  \begin{itemize}
    \item
    Ako je reč o dugmetu za određeni kurs na indeks stranici korisnika, započeće sesiju obnavljanja podrazumevanog broja pojmova koje je korisnik najdavnije naučio ili obnovio u okviru tog kursa.
    \item
    Ako se korisnik nalazi na stranici kursa i klikne na dugme \dugme{obnovi} ispod slike tog kursa, započeće sesiju obnavljanja podrazumevanog broja pojmova koje je korisnik najdavnije naučio ili obnovio u okviru tog kursa.
    \item
    Korisnik, takođe na stranici kursa, može kliknuti na lekciju koju želi da obnovi i onda klikom na dugme \dugme{obnovi} u okviru te lekcije započinje sesiju obnavljanja podrazumevanog broja  pojmova koje je korisnik najdavnije naučio ili obnovio iz te konkretne lekcije.
  \end{itemize}
  Opciono, u drugom i trećem slučaju, korisnik može klikom na padajući meni na istom dugmetu da promeni broj pojmova koje želi da obnavlja u jednoj sesiji.
  Pojmovi se prikazuju na dva različitih načina, u zavisnosti od istorije tačnosti odgovora na te pojmove.
  Načini prikaza su sledeći:
  \begin{itemize}
    \item
    Prikazace se samo pitanje ako je korisnik često odgovarao tačno na dati pojam, odnosno ako je uspešnost odgovora za taj pojam veća od neke predefinisane vrednosti.
    Korisnik unosi odgovor korišćenjem tastature i pritiskom na taster \emph{enter} ili klikom na dugme \dugme{proveri odgovor}, korisnik potvrđuje uneti odgovor.
    Ako je odgovor bio tačan, određeni pojam se više neće prikazati u okviru te sesije ni na jedan način.
    U slučaju da nije tačan, prikazuju se uneti odgovor i onaj koji je tačan kako bi korisnik mogao da vidi zbog čega je pogrešio i zapamti tačan odgovor.
    Nakon toga, korisnik nastavlja sa sesijom klikom na dugme \emph{dalje} ili pritiskom na taster \emph{enter}.
    Određeni način prikaza datog pojma na koji je unet pogrešan odgovor će se opet prikazati u toku sesije.
    \item
    Ako je korisnik često pogrešno odgovarao na određeni pojam, odnosno ako je uspešnost odgovora za taj pojam manja od neke predefinisane vrednosti, taj pojam će se u okviru sesije obnavljanja pojaviti najmanje dva puta.
    Kada se pojam prikaže prvi put, korisniku će biti prikazano pitanje i ponuđena četiri odgovora, od kojih je samo jedan tačan.
    Klikom na odgovarajući pojam ili pritiskom na taster \emph{1}, \emph{2}, \emph{3} ili \emph{4} na tastaturi, korisnik bira neki od odgovora.
    U slučaju da je odgovor bio tačan, određeni pojam se više neće prikazati na isti način u okviru te sesije.
    Ako je odgovor bio netačan, prikazuju se izabrani odgovor i onaj koji je tačan kako bi korisnik mogao da vidi zbog čega je pogrešio i zapamti tačan odgovor.
    Nakon toga, korisnik nastavlja sa sesijom klikom na \emph{dalje} ili pritiskom na taster \emph{enter}.
    Određeni način prikaza datog pojma na koji je unet pogrešan odgovor će se opet prikazati u toku sesije.
    Nakon što korisnik odgovori tačno na ovaj način prikaza, nekad u sesiji će se taj pojam prikazati još jednom, ali ovaj put biće prikazano samo pitanje.
    Korisnik unosi odgovor korišćenjem tastature i pritiskom na taster \emph{enter} ili klikom na dugme \dugme{proveri odgovor}, korisnik potvrđuje uneti odgovor.
    Ako je odgovor bio tačan, određeni pojam se više neće prikazati u okviru te sesije ni na jedan način.
    U slučaju da nije tačan, prikazuju se uneti odgovor i onaj koji je tačan kako bi korisnik mogao da vidi zbog čega je pogrešio i zapamti tačan odgovor.
    Nakon toga, korisnik nastavlja sa sesijom klikom na dugme \emph{dalje} ili pritiskom na taster \emph{enter}.
    Određeni način prikaza datog pojma na koji je unet pogrešan odgovor će se opet prikazati u toku sesije.
  \end{itemize}
  \item
  U zavisnosti od tačnosti odgovora koje je korisnik uneo, određeni pojmovi će opet biti dostupni za obnavljanje nakon određenog vremena.
  Ako je odgovor bio tačan, dati pojam će biti odložen za kasnije vreme u odnosu na onaj na koji je dat netačan odgovor.
  Kada korisnik odgovori na sve pojmove odgovarajući broj puta, sesija se završava i prikazuje se prozor kraja sesije.
  Klikom na dugme \dugme{nastavi}, započinje se nova sesija obnavljanja, a klikom na dugme \dugme{nazad na kurs}, korisniku se prikazuje stranica kursa.
\end{itemize}

\subsubsection{Alternativni tokovi}

\subsubsection{Preduslovi}
Da bi mogao da pokrene sesiju obnavljanja, korisnik treba da bude učesnik na tom kursu.
Takođe, ako za konkretni kurs ili lekciju nema pojmova za obnavljanje, neće se započeti sesija.
\subsubsection{Posledice}




\subsection{Spajalica}
\label{subsec:spajalica}

\subsubsection{Kratak opis}
Korisnik igra igru \emph{spajalica}, u kojoj mu se prikazuju kombinacije pitanja i odgovore koje on treba da poveže.
\subsubsection{Osnovni tok događaja}
\begin{itemize}
  \item
  Kada igra počne, prikazuje se prozor podeljen na devet polja.
  U poljima se nalazi pitanje ili odgovor nekog pojma.
  \item
  Korisnik klikom na dva polja ili pritiskom tastera na numeričkoj tastaturi bira neka dva od devet od ponuđenih polja.
  Cilj je izabrati pitanje i odgovor koji odgovara tom pitanju.
  \item
  Ako korisnik tačno spoji pitanje sa odgovarajućim odgovorom, na ta dva polja se pojavljuju nova pitanja ili odgovori.
  Ako nema više pitanja i odgovora za trenutnu igru, ta dva polja ostaju prazna.
  \item
  Igra se završava kada su sva polja prazna.
\end{itemize}
\subsubsection{Alternativni tokovi}

\subsubsection{Preduslovi}
Korisnik mora imati dovoljno naučenih kartica iz tog kursa ili lekcije.
\subsubsection{Posledice}




\subsection{Praćenje korisnika}
\label{subsec:pracenje-korisnika}

\subsubsection{Kratak opis}
Korisnik počinje da prati aktivnost drugog korisnika.
\subsubsection{Osnovni tok događaja}
Prijavljen korisnik koji se trenutno nalazi na stranici profila nekog drugog korisnika, klikom na dugme \dugme{prati} počinje da prati aktivnost tog korisnika.
\subsubsection{Alternativni tokovi}

\subsubsection{Preduslovi}

\subsubsection{Posledice}
%ovde nesto tipa sad kad ga prati ga vidi na rang listi na svojoj indeks strani al jebem li ga da l' to treba ovde uopste da se pise
