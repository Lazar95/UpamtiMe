\part{Arhitektura softvera}

\chapter{Kratak opis}
Dokument obrađuje pregled arhitekture aplikacije \UpamtiOnline.

\chapter{Reference}
Pogledati:
\begin{enumerate}
  \item Specifikacija slučajeva korišćenja (\ref{part:use-case}) aplikacije \UpmatiOnline.
\end{enumerate}

\chapter{Reprezentacija arhitekture}
Arhitektura sistema u dokumentu je prikazana kao serija pogleda na sistem:
pogled na slučajeve upotrebe, pogled na procese, pogled na primenjeni sistem i pogled na implementaciju.
Ovi pogledi su grafički predstavljeni kao modeli pomoću jezika UML.

\chapter{Ciljevi arhitekture}

Arhitektura je dizajnirana uzimajući u obzir sledeće:
\begin{enumerate}
  \item Korisnički interfejs je pažljivo dizajniran da bi bio jednostavan i intuitivan za korišćenje.
  \item Sve akcije koje mogu dovesti u pitanje konzistentnost i validnost podataka moraju biti programski onemogućene; na primer: na istom kursu se ne mogu naći dve kartice sa identičnim pitanjem; ne sme postojati prazna lekcija ili kurs, itd.
  \item Program mora da održava vremensku konzistentnost podataka koji se unose i obrađuju, i to pomoću mehanizama koji nisu vezani za korisnika programa.
  \item Program mora da na efikasan način radi sa velikim brojem kurseva odnosno kartica koje koristnici uče.
\end{enumerate}

\chapter{Pogled na slučajeve upoterbe}
%ako menjaš ovo promeni i u 2-.tex
Slučajevi korišćenja mogu biti podeljeni u grupe kao što je to učinjeno u \ref{ch:use-case-uvod}.

\section{Slučajevi upotrebe značajni za arhitekturu sistema}

\chapter{Pogled na logičku predstavu sistema}

\section{Pregled arhitekture -- paketi, podsistemi i slojevi}

\chapter{Pogled na procese}

\section{Procesi}
